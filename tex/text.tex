% ============================================================================ %
% Encoding: UTF-8 (žluťoučký kůň úpěl ďábelšké ódy)
% ============================================================================ %

% ============================================================================ %
\nn{Úvod}

Prešlo viac ako 120 rokov od Herzovho objavu rádiových vĺn a ich využitia Marconim
k prenosu kódovanej informácie. Za toto obdobie bolo ľudstvo schopné pokoročiť vo
využívaní rádiovej komunikácie na takú úroveň, že dnes je len málo miest na Zemi,
ktoré by neboli pokryté pozemnou rádiovou komunikáciou a takmer žiadne, ktoré by
odolali dosahu satelitnej rádiovej komunikácie.

Táto skutočnosť viedla k tomu, že dnes je možné takmer z každého miesta na Zemi
odpočúvať nejakú komunikáciu, či už je podstatná alebo nepodstatná. Táto skutočnosť
v minulosti nebola postatná, keďže pre zachytenie rádiovej komunikácie bolo potreba
buď univerzálnu a komplexnú, ťažko prenositeľnú techniku alebo špeciálne zariadenie
nastavené od výroby na jediný účel (a na relatívne úzke frekvenčné pásmo). V dôsledku
toho nebola potreba ochrany komunikácie na prvom mieste, ak vôbec bola braná v potaz.

Vývoj však ženie kupredu i techniku v tejto oblasti, a to míľovými krokmi. To čo bolo
kedysi možné len s objemným a drahým hardwarom je dnes dosiahnuteľné s ľahko prenositeľným
zariadením, ktoré sa vojde doslova do dlane. Už nie je potreba veľmi citlivých prímačov a 
amplifikáciu a filtrovanie signálu, ktorý kvôli vysokému šumu ani nebolo možné dekódovať. 
Stačí sa dostaviť do rozumnej vzdialenosti, a ani konfigurovateľná filtrácia ``za behu''
nerobí modernej technike problémy.

A čo viac, komplexná a rýchla komunikácia, vyžadujúca logiku stavových automatov na čipe, 
je dnes nahraditeľná rýchlymi procesormi a ich výpočetnou kapacitou či vhodnými, na mieste
reprogramovateľnými, čipmi. Je tak skutočne možné dekódovať takmer akúkoľvek komunikáciu
a teda i na ňu útočiť, čo otvára mnohé otázky ohľadom zabezpečenia rádiovej komunikácie v
našej spoločnosti.

Táto práca sa zaoberá problematikou vyhľadávania slabých miest spôsobených nedostatočným
zabezpečením alebo designovou chybou v návrhu komunikačných prvkov. V Sekcii \ref{sec:elmag} a \ref{sec:radio} rozoberá fyzikálne princípy a základy rádiovej komunikácie, Sekcia \ref{sec:sdr} sa venuje univerzálnemu zariadeniu využívajúcemu software pre rádiovú komunikáciu -- Software Defined Radio (SDR). Sekcia \ref{sec:security} ďalej skúma bezpečnosť v danej oblasti a Sekcia \ref{sec:softtools} prezentuje využiteľný software a nástroje pre analýzu a spracovanie zachytenej komunikácie. Nasledujúce sekcie prezentujú návrh a zostavenie zariadenia umožňujúceho skúmať bezpečnosť v teréne, a to na základe existujúcich komponentov (Sekcia \ref{sec:design}), tak i novo vytvorených (Sekcia \ref{sec:attacks}). Posledná Sekcia \ref{sec:testing} skúma možnosti takéhoto zariadenia -- v testovacom i reálnom prostredí.


% ============================================================================ %
\cast{Teoretická časť}

\n{1}{Elektromagnetické spektrum} \label{sec:elmag}

Elektromagnetická sila je jednou zo štyroch základných síl, ktoré boli fyzici schopní
objaviť a dopodrobna matematicky popísať. Niet preto divu, že využitie tejto sily nás
sprevádza každý deň.

Už James Clerk Maxwell v klasickej teórii elektromagnetizmu, ktorá vznikla najmä z jeho práce, 
prepovedal vznik elektromagnetických vĺn, ktoré sa šíria priestorom \cite{electromag}. V tej
dobe predstava zahrňovala hlavne svetlo, ale matematicky dávali zmysel i iné frekvencie.

\obr{Rozdelenie elektromagnetického spektra, \b{f} predstavuje frekcencie a $\bm{\lambda}$ vlnové dĺžky \cite{nasaspectrum}.}{img:elmagspectrum}{0.9}{graphics/elmag_spectrum.png}

Z dnešného hľadiska môžeme elektromagnetické spektrum rozdeliť do nasledujúcich kategórií (viď Obr. \ref{img:elmagspectrum}):
\begin{itemize}
\item pásmo rádiových frekvencií
\item infračervené pásmo
\item viditeľné svetlo
\item ultrafialové pásmo
\item rontgenové pásmo
\item pásmo gamma
\item pásmo kozmického žiarenia
\end{itemize}

Frekcencie nad ultrafialovým žiarením (vrátene jeho časti) radíme do kategórie \i{ionizujúceho} žiarenia, pretože ich vplyvom dochádza k zmene genetickej informácie a rozpadu bunečného života. Tieto frekvencie na Zemi filtrujú vrstvy atmosféry a umožňujú tak život na našej planéte.

Nižšie frekvencie okolo seba vidíme každý deň. Ich vplyv na život je minimálny, prevážne sa premieňajú na teplo. Špeciálne rádiové frekvencie majú schopnosť do istej miery prenikať predmetmi a práve z tohoto dôvodu sú vhodné pre komunikáciu na väčšie vzdialenosti. Ďalšie detaily v tomto smere poskytne nasledujúca kapitola.

\n{1}{Rádiový prenos dát} \label{sec:radio}

\n{2}{Využívané frekvencie a legislatíva}

Po úvodnom objave existencie a možnostiach využitia elektromagnetických vĺn, špeciálne v rádiovej oblasti, nastal veľký boom v ich využívaní, ktorý viedol až k zavedeniu noriem ich používania a prideľovana frekvenčných pásiem pre isté spôsoby ich využitia.

\obr{Rozdelenie rádového spektra podľa ITU, \b{f} predstavuje frekcencie a $\bm{\lambda}$ vlnové dĺžky \cite{nasaspectrum}.}{img:radiospectrum}{0.7}{graphics/radio_spectrum.png}

Rádiové spektrum tvoria vlny so širokým rozpätím frekvencií približne do 100 GHz (viď Obr. \ref{img:radiospectrum}) a môžeme ich rozdeliť do nasledujúcich kategórií \cite{itu}:
\begin{itemize}
\item pásmo veľmi nízkych frekvencií (VLF) -- využitie v komunikácii s ponorkami, v komunikácii v podzemních baniach
\item pásmo nízkych frekvencií (LF) -- využitie v navigácii, synchronizácii časových signálov, pre amatérske rádio a identifikáciu na rádiovej frekvencii(RFID)
\item pásmo stredných frekvencií (MF) -- využitie pre vysielanie rádia v oblasti amplitúdovej modulácie (AM), lavínové záchranné vysielače,, pre amatérske rádio
\item pásmo vysokých frekvencií (HF) -- využitie pre vbezkontaktné platby a komunikáciu blízkeho poľa (NFC), leteckú komunikáciu ``za horizont'', komunikáciu na mori, bezdrôtové domáce telefóny
\item pásmo veľmi vysokých frekvencií (VHF) -- využitie pre vysielanie rádia v oblasti frekvenčnej modulácie (FM), televízne prenosy, amatérske rádio, bezpečnostné zložky (polícia, hasiči, a pod.), komunikáciu informácií o počasí z meteorologických staníc
\item pásma ultra (UHF) a super (SHF) vysokých frekvencií -- mikrovlnná komunikácia -- bezdrôtovú lokálnu sieť (WiFi), BlueTooth, ZigBee, LoRa, a pod., mikrovlnný ohrev jedla, satelitnú navigáciu (GPS, Galileo, Glonass, Beidou), satelitná telefónna komunikácia, satelitné vysielanie, a pod.
\item pásmo extra vysokých frekvencií (EHF) -- rádio astronómia, mikrovlnné zbrane, a pod.
\end{itemize}

Z uvedených pásiem sa táto práca bude prevažne orientovať na rozmedzie od vysokých frekvencií (HF) až po super vysoké frekvencie (SHF).

\n{2}{Štandardné komponenty}

\n{2}{Abstraktné vrstvy a protokoly}

\n{2}{Rádiové a bezdrôtové systémy}

\n{1}{Software Defined Radio (SDR)} \label{sec:sdr}

\n{2}{História SDR}

\n{2}{Komponenty SDR}

\n{2}{Súčasný stav SDR}

\n{1}{Bezpečnosť rádiových systémov} \label{sec:security}

\n{2}{Analýza bezpečnosti rádiových systémov}

\n{2}{Silné, slabé miesta a problémy prevedenia útoku}

\n{2}{Kategórie útokov}

\n{2}{Využitie SDR pri útokoch}

\n{1}{Software a nástroje} \label{sec:softtools}

\n{2}{Analýza frekvenčného spektra}

\n{2}{Detekcia typu komunikácie a použitých protokolov}

\n{2}{Tvorba packetov}

\n{2}{Tvorba sledu komunikácie}

\cast{Praktická časť}

\n{1}{Návrh systému s SDR} \label{sec:design}

\n{2}{Analýza požiadavkov pre návrh systému}

\n{2}{Výber hardware komponentov}

\n{2}{Výber software komponentov}

\n{2}{Zostavenie systému a nastavenie software}

\n{1}{Návrh detekcie možných vektorov útoku} \label{sec:attacks}

\n{2}{Automatická analýza rádiovej komunikácie}

\n{2}{Detekcia vybraných protokolov a zachytávanie komunikácie}

\n{2}{Zostavenie možných útokov na základe komunikácie}

\n{2}{Prevedenie útoku}

\n{1}{Testovanie systému} \label{sec:testing}

\n{2}{Testovanie komponent systému}

\n{2}{Testovanie systému imitáciou reálneho cieľa}

\n{2}{Nasadenie systému v reálnej prevádzke}

% ============================================================================ %
\nn{Záver}


% ============================================================================ %
